\documentclass[usepdftitle=false]{beamer}

\usepackage[frenchb]{babel}
\usepackage[T1]{fontenc}
\usepackage[utf8]{inputenc}
\usepackage{graphicx}
\usepackage{datetime}
\usepackage{eurosym}
\usepackage[]{url}
\usepackage[babel=true]{csquotes}
\hypersetup{
pdfauthor={Thibaud Duhautbout - Rémy Huet},
pdftitle={Formation Picaosft : La gestion de version avec Git},
pdfsubject={Formation niveau 1 : les bases},
pdfkeywords={git, gestion de version, VCS},
pdfproducer={Latex},
}

\newdateformat{nombres}{\THEDAY-\THEMONTH-\THEYEAR}

\title[Formation Git\_v1]{\today \\ Formation Picasoft : La gestion de version avec Git (niveau 1)}
\titlegraphic{\includegraphics[scale=.1]{picasoft_logo.png}}
\author[T. Duhautbout - R. Huet]{Thibaud {\sc Duhautbout} \\ Rémy {\sc Huet}}
\institute[Picasoft]{Association Picasoft}
\date\today

\usetheme{AnnArbor}
\usecolortheme{crane}

\AtBeginSection[]
{
	\begin{frame}
		\tableofcontents[currentsection, hideothersubsections]
	\end{frame}
}

\begin{document}

\begin{frame}
	\titlepage 
\end{frame}

\section{Introduction}

\begin{frame}{Pourquoi la gestion de version ?}
\end{frame}

\begin{frame}{Les différents logiciels de version}
\end{frame}

\begin{frame}{Petite histoire de git\ldots}
\end{frame}

\section{Concepts de base}

\subsection{Configuration et initialisation}

\begin{frame}{git config}
	git config -- Configuration de l'identité de l'utilisateur

	git init -- Initialisation du repo
\end{frame}

\subsection{\'Etat du repo local}

\begin{frame}{git status}
\end{frame}

\subsection{Ajouter une version}

\begin{frame}{git add -- reset -- commit}
	Staging area etc
\end{frame}

\subsection{Voir l'historique}

\begin{frame}{git log}
\end{frame}

\begin{frame}{git diff}
\end{frame}

\section{Concepts avancés}

\subsection{Le HEAD}

\begin{frame}{Le HEAD}
\end{frame}

\subsection{Enregistrer les modifications locales}

\begin{frame}{git stash}
\end{frame}

\subsection{Changer de version}

\begin{frame}{git checkout}
\end{frame}

\subsection{Annuler les modifications sur un fichier précis}

\begin{frame}{git checkout -- file}
\end{frame}

\section{Les remotes}

\subsection{Principe et application avec Gitlab}

\begin{frame}{Gitlab}
	Création d'un repo
\end{frame}

\subsection{Récupérer les ajouts distants}

\begin{frame}{git clone -- git pull}
	cloner le repo de la présentation (en HTTPS)
\end{frame}

\subsection{Envoyer des modifications}

\begin{frame}{git push}
	Sur repo perso
\end{frame}

\end{document}
