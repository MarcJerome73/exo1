\documentclass[usepdftitle=false]{beamer}

\usepackage[frenchb]{babel}
\usepackage[T1]{fontenc}
\usepackage[utf8]{inputenc}
\usepackage{graphicx}
\usepackage{datetime}
\usepackage{eurosym}
\usepackage[]{url}
\usepackage[babel=true]{csquotes}
\usepackage{listings}
\hypersetup{
pdfauthor={Thibaud Duhautbout - Rémy Huet},
pdftitle={Formation Picaosft : La gestion de version avec Git},
pdfsubject={Formation niveau 1 : les bases},
pdfkeywords={git, gestion de version, VCS},
pdfproducer={Latex},
}

\beamertemplatenavigationsymbolsempty
\setbeamercolor{orangebox}{bg=orange,fg=black}

\newdateformat{nombres}{\THEDAY-\THEMONTH-\THEYEAR}

\title[Formation Git\_v1]{\today \\ Formation Picasoft : La gestion de version avec Git (niveau 1)}
\titlegraphic{\includegraphics[scale=.1]{picasoft_logo.png}}
\author[T. Duhautbout - R. Huet]{Thibaud {\sc Duhautbout} \\ Rémy {\sc Huet}}
\institute[Picasoft]{Association Picasoft}
\date\today

\usetheme{AnnArbor}
\usecolortheme{crane}

\AtBeginSection[]
{
	\begin{frame}
		\tableofcontents[currentsection, hideothersubsections]
	\end{frame}
}

\begin{document}

\begin{frame}
	\titlepage 
\end{frame}

\section{Introduction}

\begin{frame}{Pourquoi la gestion de version ?}
Pour tout le monde :
\begin{itemize}
\item sauvegarde incrémentale du travail
\item suivi des modifications et retour en arrière
\item partage des modifications avec d'autres personnes
\end{itemize}

\medskip

Pour les développeurs :
\begin{itemize}
\item centralisation des sources
\item collaboration simplifiée
\item possibilité de maintenir plusieurs versions
\item ...
\end{itemize}
\end{frame}

\begin{frame}{Les différents logiciels de version}
\includegraphics[height=2cm]{./imgs/logo_git.png}
\hfill
\includegraphics[height=2cm]{./imgs/logo_svn.png}
\hfill
\includegraphics[height=2cm]{./imgs/logo_mercurial.png}

\bigskip

\centering
Et plein d'autres !

37 systèmes recensés sur Wikipedia

(\url{https://en.wikipedia.org/wiki/Comparison_of_version_control_software})
\end{frame}

\begin{frame}{Petite histoire de git\ldots}

\begin{center}
\includegraphics[height=2cm]{./imgs/logo_git.png}
\end{center}

\begin{itemize}
\item Créé en 2005 par les développeurs du noyau Linux
\item Système de gestion de version distribué
\item Rapide
\item Possibilité de développements non-linéaires (branches)
\item Popularité grandissante chez les développeurs (GitHub, GitLab)
\end{itemize}
\end{frame}

\begin{frame}
\begin{beamercolorbox}[sep=5pt,center,rounded=true,shadow=true]{orangebox}
\Large
\textsc{On passe à la partie pratique !}
\end{beamercolorbox}
\begin{center}
\includegraphics[height=3cm]{./imgs/warning.jpg}
\end{center}

\centering
Dans la suite, on considère que git est installé pour toutes les machines !

\medskip

\centering
En cas de problème, n'hésitez pas à demander de l'aide aux gentils animateurs munis d'une pancarte \enquote{HELP} :)

\begin{block}{}
\centering
On ouvre un terminal ou Git Bash !
\end{block}
\end{frame}

\section{Concepts de base}

\subsection{Configuration et initialisation}

\begin{frame}[fragile]{Configuration de git et du repo}

\begin{block}{Création d'un nouveau répertoire de travail}
\verb+$ mkdir formation_git+ \\
\verb+$ cd formation_git+
\end{block}

\begin{block}{Configuration de l'identité de l'utilisateur}
Permet d'identifier l'auteur des mises à jour \\
\verb+$ git config --global/local user.name "<prénom nom>"+ \\
\verb+$ git config --global/local user.email "<adresse email>"+ \\
\verb+global+ : configuration au niveau du système \\
\verb+local+ : configuration au niveau du répertoire courant 
\end{block}

\begin{block}{Initialisation du repo git}
\verb+$ git init+
\end{block}

\end{frame}

\subsection{\'Etat du repo local}

\begin{frame}{git status}
\end{frame}

\subsection{Ajouter une version}

\begin{frame}{git add -- reset -- commit}
	Staging area etc
\end{frame}

\subsection{Voir l'historique}

\begin{frame}{git log}
\end{frame}

\begin{frame}{git diff}
\end{frame}

\section{Concepts avancés}

\subsection{Le HEAD}

\begin{frame}{Le HEAD}
\end{frame}

\subsection{Enregistrer les modifications locales}

\begin{frame}{git stash}
\end{frame}

\subsection{Changer de version}

\begin{frame}{git checkout}
\end{frame}

\subsection{Annuler les modifications sur un fichier précis}

\begin{frame}{git checkout -- file}
\end{frame}

\section{Les remotes}

\subsection{Principe et application avec Gitlab}

\begin{frame}{Gitlab}
	Création d'un repo
\end{frame}

\subsection{Récupérer les ajouts distants}

\begin{frame}{git clone -- git pull}
	cloner le repo de la présentation (en HTTPS)
\end{frame}

\subsection{Envoyer des modifications}

\begin{frame}{git push}
	Sur repo perso
\end{frame}

\end{document}
